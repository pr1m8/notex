
\documentclass{article}
\usepackage[headheight=20pt, margin=1.0in, top=1.2in]{geometry}
\usepackage{amsmath, amssymb, amsthm, thmtools, tcolorbox, array, graphicx, makeidx, cancel, multirow, fancyhdr, xypic, color, nicefrac, rotating, multicol, caption, subcaption, xcolor, tikz, tikz-3dplot, tikz-cd, pgfplots, import, enumitem, calc, booktabs, wrapfig, siunitx, hyperref,float}
\hypersetup{colorlinks=true,linkcolor=blue}
\usepackage[all]{xy}
\usepackage{esint}
\setlength{\parindent}{0in}
\sisetup{per-mode = symbol}
\usetikzlibrary{calc,arrows,svg.path,decorations.markings,patterns,matrix,3d,fit}
\usepgfplotslibrary{groupplots}
\pgfplotsset{compat=newest}
\newtcolorbox{mydefbox}[2][]{colback=red!5!white,colframe=red!75!black,fonttitle=\bfseries,title=#2,#1}
\newtcolorbox{mythmbox}[2][]{colback=gray!5!white,colframe=gray!75!black,fonttitle=\bfseries,title=#2,#1}
\newtcolorbox{myexamplebox}[2][]{colback=green!5!white,colframe=green!75!black,fonttitle=\bfseries,title=#2,#1}
\newtcolorbox{mypropbox}[2][]{colback=blue!5!white,colframe=blue!75!black,fonttitle=\bfseries,title=#2,#1}
\declaretheoremstyle[headfont=\color{blue}\normalfont\bfseries,]{colored}
\theoremstyle{definition}
\newtheorem{theorem}{Theorem}
\newtheorem{corollary}[theorem]{Corollary}
\newtheorem{lemma}[theorem]{Lemma}
\newtheorem{proposition}[theorem]{Proposition}
\newtheorem{problem}[theorem]{Problem}
\newtheorem{definition}[theorem]{Definition}
\newtheorem{exercise}[theorem]{Exercise}
\newtheorem{example}[theorem]{Example}
\newtheorem{solution}[theorem]{Solution}
\newtheorem*{thm}{Theorem}
\newtheorem*{lem}{Lemma}
\newtheorem*{prob}{Problem}
\newtheorem*{exer}{Exercise}
\newtheorem*{prop}{Proposition}
\def\R{\mathbb{R}}
\def\F{\mathbb{F}}
\def\Q{\mathbb{Q}}
\def\C{\mathbb{C}}
\def\N{\mathbb{N}}
\def\Z{\mathbb{Z}}
\def\Ra{\Rightarrow}
\def\e{\epsilon}
\newcommand{\typo}[1]{{\color{red}{#1}}}
\newcommand\thedate{\today}
\newcommand{\mb}{\textbf}
\newcommand{\norm}[2]{\|{#1}\|_{#2}}
\newcommand{\normm}[1]{\|#1\|}
\newcommand{\mat}[1]{\begin{bmatrix} #1 \end{bmatrix}}
\newcommand{\eqtext}[1]{\hspace{3mm} \text{#1} \hspace{3mm}}
\newcommand{\set}[1]{\{#1\}}
\newcommand{\inte}{\textrm{int}}
\newcommand{\ra}{\rightarrow}
\newcommand{\minv}{^{-1}}
\newcommand{\tx}[1]{\text{ {#1} }}
\newcommand{\abs}[1]{|#1|}
\newcommand{\mc}[1]{\mathcal{#1}}
\newcommand{\uniflim}{\mathop{\mathrm{unif\lim}}}
\newcommand{\notimplies}{\mathrel{{\ooalign{\hidewidth$\not\phantom{=}$\hidewidth\cr$\implies$}}}}
\pagestyle{fancy}
\fancyhf{}
\fancyhead[L]{Title of the Document}
\fancyhead[C]{}
\fancyhead[R]{\thepage}
\fancyfoot[L]{}
\fancyfoot[C]{}
\fancyfoot[R]{}
\renewcommand{\headrulewidth}{0.4pt}
\renewcommand{\footrulewidth}{0.4pt}
\numberwithin{equation}{section}
% Increase spacing between paragraphs
\setlength{\parskip}{1em}
% Increase spacing before and after sections
\usepackage{titlesec}
\titlespacing*{\section}{0pt}{3ex plus 1ex minus .2ex}{2ex plus .2ex}
\titlespacing*{\subsection}{0pt}{2ex plus 1ex minus .2ex}{1ex plus .2ex}
\titlespacing*{\subsubsection}{0pt}{1ex plus 1ex minus .2ex}{1ex plus .2ex}
\title{\textbf{Title of the Document}}
\author{Author Name}
\date{\today}
\begin{document}
\maketitle
\tableofcontents
\newpage
\section*{2.6 Exercises}

\begin{enumerate}
    \item Let $(X,d)$ be a metric space and $S \subset X$. Show that $\partial S = \emptyset$ if and only if $S^c$ is open.
    \item Show that for an arbitrary choice of $a,b,r \in \mathbb{R}$, the closed disk $D = \{(x,y) \in \mathbb{R}^2 \mid (x-a)^2 + (y-b)^2 \leq r^2\}$ is a bounded set in $\mathbb{R}^2$.
    \item Let $(X,d)$ be a metric space and let $x,y \in X$. Show that if $d(x,y) < \epsilon$ for every $\epsilon > 0$, then $x = y$.
\end{enumerate}

\bigskip
\noindent\textbf{2.6 (i)}

Assume $\partial S = \emptyset$. Then $\forall x \in S$, $\exists \epsilon > 0$ : $B_\epsilon(x) \subseteq S$.
Then by $x \in S^{\text{int}}$, $\exists \epsilon > 0\ \ B_\epsilon(x) \subseteq S$.
However, by $x \in \partial S$, this value of $\epsilon > 0$ implies $B_\epsilon(x) \cap S^c \neq \emptyset \implies B_{\frac{\epsilon}{2}}(x) \not\subseteq S$, which is a contradiction.
impliing our assumption that $x \in \partial S \cap S^{\text{int}}$ must be false and 
$\partial S \cap S^{\text{int}} = \emptyset$.

\bigskip
\noindent\textbf{2.6 (ii)}

A set $S$ is bounded iff $\exists  M \in \mathbb{R}^+$ : $\forall x, y \in  S$ 
$d(x,y) \leq M$ 

Let $a,b,r \in \mathbb{R}$.

$S = \{ (x,y) \in \mathbb{R}^2 \mid (x-a)^2 + (y-b)^2 \leq r^2 \}$ 

$\implies x^2 - 2ax + a^2+ y^2 - 2yb + b^2 \leq r^2$
$\implies x^2 - 2ax + y^2 - 2yb + r^2 \leq r^2 - a^2 - b^2$ 

$\implies x^2 + y^2 \leq r^2 - a^2 - b^2 + 2 a x 2 yb $

need to show $x^2$ is bounded

$\cdot   (x-a)^2 \leq r^2$

$\implies |x-a|\leq| r| $

$\implies |x-a| \leq | r| + | a| $

$\implies|x|=| x - a + a | \leq | x - a| + | a| \leq r + | a|$

\begin{align*}
\implies |y| & \leq r + |a| \\
\implies y^2 & \leq (r + |a|)^2
\end{align*}

Same for \ y,\ y^2 \leq (r+ |b|)^2 

$\forall z = (x,y) \in D^2_{a,b} $

$\|z\| = \sqrt{x^2 + y^2}$

$\leq \sqrt{(r+|a|)^2 + (r+|b|)^2}$

Thus if $M = \sqrt{(r+|a|)^2 + (r + |b|)^2} , \text{the bound holds.}$

$ \# 1S \ named \ boundedne \ =>\ distance boundedness. $

Let 
$x = (x_1, x_2), \ y = (y_1, y_2) \in D^2_{a,b} $
$ z_1 = {x_1,y_1} $ ,
$ (x_2 - a)^2 + (x_2 - b)^2 = r^2 $
$\Rightarrow d(z_1,(a,b)) = \sqrt{(x_2 - a)^2 + (x_2 - b)^2} \le r $
$\Rightarrow d(x,y) \leq d(x_1,(a,b)) + d(y_1,(a,b)) $

$ = \sqrt{(x_1 - a)^2 +(x_1 - b)^2} + \sqrt{(y_1 - a)^2 +(y_1 - b)^2} $ 
$ \le r + r = 2r. $
\begin{enumerate}
    \item[(iii)] Suppose that \(x \neq y\). Then \(d(x, y) \neq 0\). Thus, if we choose \(\epsilon = d(x, y)\), then \(\epsilon > 0\) but \(d(x, y) \geq \epsilon\) (contradiction).

    \textit{(Contradiction)} Suppose \(x = y\) and so \(d(x, y) = 0\).
    $
    \text{Choose } \epsilon > 0 \text{ so that } \epsilon = d(x, y)/2. \text{ Then we must have } d(x, y) < \epsilon = \frac{d(x, y)}{2},
    $
    which is a contradiction, as this implies if \(d(x, y) \leq 0\) then \(d(x, y) = 0 = \epsilon = \frac{\epsilon}{2}\).
    $
    \Rightarrow 3\epsilon < \frac{\epsilon}{2} \Rightarrow 3 \cdot 2 < \epsilon
    $
    Thus \(x \neq y\).

    \item[(iv)] Let \((V, \lVert \cdot \rVert)\) be a normed vector space. Then let \(r > 0\) and \(x \in V\). Then
    $
    B_r(x) = \{ u \in V \mid d(x, u) < r \}
    $
    $
    B_{\epsilon + \lVert x \rVert}(0) = \{ u \in V \mid d(0, u) < \epsilon + \lVert x \rVert \}
    $
    $
    \text{Let } y \in B_r(x).
    $
    $
    d(0, y) \leq d(0, x) + d(x, y)
    $
    $
    \leq \lVert x \rVert + r
    $
    $
    \Rightarrow B_r(x) \subseteq B_{\epsilon + \lVert x \rVert}(0).
    $

    \item[(v)] Suppose \(S\) is bounded. Then \(\exists M \in \mathbb{R} : \forall x\in S \lVert x \rVert \leq M\). (Equal to \(\exists M \in \mathbb{R} : \forall x \in V x \in B_M(0)\))
\end{enumerate}

\begin{figure}
    \centering
    % The figure depicting the relationship between the normed vector space, the balls, and the distances
    % since we cannot directly draw it using LaTeX, consider adding the sketch or explaining the relationships.
    \caption{Sketch of the normed vector space \( (V, \lVert \cdot \rVert) \), its balls \( B_r(x) \) and \( B_{\epsilon + \lVert x \rVert}(0) \).}
\end{figure}\end{document}