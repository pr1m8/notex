
\documentclass{article}
\usepackage[headheight=20pt, margin=1.0in, top=1.2in]{geometry}
\usepackage{amsmath, amssymb, amsthm, thmtools, tcolorbox, array, graphicx, makeidx, cancel, multirow, fancyhdr, xypic, color, nicefrac, rotating, multicol, caption, subcaption, xcolor, tikz, tikz-3dplot, tikz-cd, pgfplots, import, enumitem, calc, booktabs, wrapfig, siunitx, hyperref,float}
\hypersetup{colorlinks=true,linkcolor=blue}
\usepackage[all]{xy}
\usepackage{esint}
\setlength{\parindent}{0in}
\sisetup{per-mode = symbol}
\usetikzlibrary{calc,arrows,svg.path,decorations.markings,patterns,matrix,3d,fit}
\usepgfplotslibrary{groupplots}
\pgfplotsset{compat=newest}
\newtcolorbox{mydefbox}[2][]{colback=red!5!white,colframe=red!75!black,fonttitle=\bfseries,title=#2,#1}
\newtcolorbox{mythmbox}[2][]{colback=gray!5!white,colframe=gray!75!black,fonttitle=\bfseries,title=#2,#1}
\newtcolorbox{myexamplebox}[2][]{colback=green!5!white,colframe=green!75!black,fonttitle=\bfseries,title=#2,#1}
\newtcolorbox{mypropbox}[2][]{colback=blue!5!white,colframe=blue!75!black,fonttitle=\bfseries,title=#2,#1}
\declaretheoremstyle[headfont=\color{blue}\normalfont\bfseries,]{colored}
\theoremstyle{definition}
\newtheorem{theorem}{Theorem}
\newtheorem{corollary}[theorem]{Corollary}
\newtheorem{lemma}[theorem]{Lemma}
\newtheorem{proposition}[theorem]{Proposition}
\newtheorem{problem}[theorem]{Problem}
\newtheorem{definition}[theorem]{Definition}
\newtheorem{exercise}[theorem]{Exercise}
\newtheorem{example}[theorem]{Example}
\newtheorem{solution}[theorem]{Solution}
\newtheorem*{thm}{Theorem}
\newtheorem*{lem}{Lemma}
\newtheorem*{prob}{Problem}
\newtheorem*{exer}{Exercise}
\newtheorem*{prop}{Proposition}
\def\R{\mathbb{R}}
\def\F{\mathbb{F}}
\def\Q{\mathbb{Q}}
\def\C{\mathbb{C}}
\def\N{\mathbb{N}}
\def\Z{\mathbb{Z}}
\def\Ra{\Rightarrow}
\def\e{\epsilon}
\newcommand{\typo}[1]{{\color{red}{#1}}}
\newcommand\thedate{\today}
\newcommand{\mb}{\textbf}
\newcommand{\norm}[2]{\|{#1}\|_{#2}}
\newcommand{\normm}[1]{\|#1\|}
\newcommand{\mat}[1]{\begin{bmatrix} #1 \end{bmatrix}}
\newcommand{\eqtext}[1]{\hspace{3mm} \text{#1} \hspace{3mm}}
\newcommand{\set}[1]{\{#1\}}
\newcommand{\inte}{\textrm{int}}
\newcommand{\ra}{\rightarrow}
\newcommand{\minv}{^{-1}}
\newcommand{\tx}[1]{\text{ {#1} }}
\newcommand{\abs}[1]{|#1|}
\newcommand{\mc}[1]{\mathcal{#1}}
\newcommand{\uniflim}{\mathop{\mathrm{unif\lim}}}
\newcommand{\notimplies}{\mathrel{{\ooalign{\hidewidth$\not\phantom{=}$\hidewidth\cr$\implies$}}}}
\pagestyle{fancy}
\fancyhf{}
\fancyhead[L]{Title of the Document}
\fancyhead[C]{}
\fancyhead[R]{\thepage}
\fancyfoot[L]{}
\fancyfoot[C]{}
\fancyfoot[R]{}
\renewcommand{\headrulewidth}{0.4pt}
\renewcommand{\footrulewidth}{0.4pt}
\numberwithin{equation}{section}
% Increase spacing between paragraphs
\setlength{\parskip}{1em}
% Increase spacing before and after sections
\usepackage{titlesec}
\titlespacing*{\section}{0pt}{3ex plus 1ex minus .2ex}{2ex plus .2ex}
\titlespacing*{\subsection}{0pt}{2ex plus 1ex minus .2ex}{1ex plus .2ex}
\titlespacing*{\subsubsection}{0pt}{1ex plus 1ex minus .2ex}{1ex plus .2ex}
\title{\textbf{Title of the Document}}
\author{Author Name}
\date{\today}
\begin{document}
\maketitle
\tableofcontents
\newpage
\section{Exercises}
\begin{itemize}
    \item[2.1.] Let \((X,d)\) be a metric space and \(S \subset X\). Show that \( \partial S^n \subseteq S^*\).

    \item[2.2.] Show that for an arbitrary choice of \(a, b, r\) in \(\mathbb{R}\), the closed disk \((x-a)^2 + (y-b)^2 \leq r^2\) is a bounded set in \(\mathbb{R}^2\).

    \item[2.3.] Let \((X, d)\) be a metric space and for \(x, y \in X\), show that if \(d(x, y) < \epsilon\) for every \(\epsilon > 0\), then \(x = y\).

\end{itemize}

\begin{problem}
\textbf{(2.1)} Assume \( \partial S^c \subseteq \exists \epsilon_1 \in \mathbb{R}^0 \mid x \in S^n S^* \subseteq S^n \).

Then by \( x \in S^{int} \implies \exists \epsilon > 0 : B_{\epsilon}(x) \subseteq S \).

However, by \( x \in \partial S \), this value of \(\epsilon > 0\) implies $ B_{\epsilon}(x) \cap S = \emptyset \implies B_{\epsilon}(x) \nsubseteq S $
which is a contradiction, implying our assumption that \( x \in \partial S \cap S^{int} \) must be false and
$ \partial S \cap S^{int} = \emptyset $
\end{problem}

\begin{problem}
\textbf{(2.2)} A set \( S \) is bounded iff \( \exists M \in \mathbb{R}^+ : \forall x, y \in S \, d(x, y) \leq M \)

Let \( a, b, r \in \mathbb{R}\).

$\delta := \{ (u, v) \in \mathbb{R} \mid (x - a)^2 + (y - b)^2 \leq r^2 \}$

$
\Rightarrow x^2 - 2ax + a^2 + y^2 - 2yb + b^2 \leq r^2
$
$
\Rightarrow x^2 - 2ax + y^2 - 2yb \leq r^2 - a^2 - b^2
$
$
\Rightarrow x^2 + y^2 \leq r^2 - a^2 - b^2 + 2ax + 2yb
$

Need to show \( x^2 \) is bounded
$ \cdot (x - a)^2 \leq r^2 $
$
\Rightarrow \vert x - a \vert \leq \vert r \vert
$
$
\Rightarrow \vert x - a \vert + a \leq \vert r \vert + |a|
$
$
\Rightarrow \vert x \vert = \vert x - a + a \vert \leq \vert x - a \vert + \vert a \vert \leq R + |a|
$

\begin{align*}
&\quad\quad\quad\quad\quad\quad\quad\quad\quad\quad\quad\quad\quad \{ b\\ \quad \quad\quad\quad \bigcirc \\
a
\end{align*}

\Rightarrow |y| \leq r + |a| \\
\Rightarrow x^2 \leq (r + |a|)^2 \$10pt]

\text{Same for } y, \ y^2 \leq (r + |b|)^2 \$10pt]

\forall z = (x, y) \in D^2_{a,b},\$10pt]

\|z\| = \sqrt{x^2 + y^2}\\
\quad \quad \quad \leq \sqrt{(r + |a|)^2 + (r + |b|)^2} \$10pt]

\text{Thus if } M = \sqrt{(r + |a|)^2 + (r + |b|)^2}, \ \text{the band holds.}\$10pt]

\underline{\#15 \ \text{Normed boundedness} = \text{distance boundedness.}}\$10pt]

\text{Let } x = (x_1, x_2), \ y = (y_1, y_2) \in D_{a, b},\$10pt]

z_1 \in \{x_1, y_1\}\$10pt]

(x_2 - b)^2 + (x_2 - b)^2 = r^2\\
\Rightarrow d((x, a), b) = \parallel(x_1 - a)^2 + (x_2 - b)^2 \parallel \leq r \$10pt]

\Rightarrow d(x, y) \leq d(x, (a, b)) + d((a, b), y)\\
= \sqrt{(x_1 - a)^2 + (x_2 - b)^2} + \sqrt{(y_1 - a)^2 + (y_2 - b)^2}\\
\Rightarrow r + r = 2r\\

\begin{enumerate}
\item[(iii)] Suppose that $x \neq y$. Then $d(x, y) \neq 0$. Thus if we choose $\varepsilon = d(x, y) \Rightarrow \varepsilon > 0$ but $d(x, y) \geq \varepsilon$. (contradiction).
\newline

\item[(contradiction)] Suppose $x \neq y$ and so $d(x, y) \neq 0$. \\ Choose $\varepsilon > 0$ so that $\varepsilon = d(x, y)$. Then we must have:

$
 d(x, y) < \varepsilon = \frac{d(x,y)}{2}, \text{ which is a contradiction, as this implies }
 $

if $d(x,y) \leq \frac{d(x,y)}{2} \Rightarrow d(x,y) = \varepsilon < \frac{\varepsilon}{2} = \frac{d(x,y)}{2}$ 
$
 \Rightarrow 1>2
 $

Thus $x = y$

\item[(iv)] Let $(V, \| \cdot \|)$ be a normed vsp. \\ Then let $r > 0$ and $x \in V$. Then

$ 
B_r(x) = \{u \in V \,| \, d(x, u) < r\}
$
$
B_{\lambda \| \cdot \| }(0) = \{v \in V \,| \, d(0,v) < r + \| x \|\}
$
$
\includegraphics[width=0.5\textwidth]{diagram.png}
$

Let $y \in B_r(x)$.
$
 d(0, y) \le d(0, x) + d(x, y)
$

$
 \le \| x \| + r
 $

$
 \Rightarrow B_r(x) \subseteq B_{r + \| x \| }(0)
 $

\item[(v)] Suppose $\mathcal{S}$ is bounded. Then $\exists M \in \mathbb{R}: \forall x \in \mathcal{S} \| x\| \leq M$. 

(Equal to $\exists M \in \mathbb{R}: \forall x \in \mathcal{S} x \in B_M(0)$
\end{enumerate}\end{document}