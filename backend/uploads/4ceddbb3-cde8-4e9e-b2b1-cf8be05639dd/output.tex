
\documentclass{article}
\usepackage[headheight=20pt, margin=1.0in, top=1.2in]{geometry}
\usepackage{amsmath, amssymb, amsthm, thmtools, tcolorbox, array, graphicx, makeidx, cancel, multirow, fancyhdr, xypic, color, nicefrac, rotating, multicol, caption, subcaption, xcolor, tikz, tikz-3dplot, tikz-cd, pgfplots, import, enumitem, calc, booktabs, wrapfig, siunitx, hyperref,float}
\hypersetup{colorlinks=true,linkcolor=blue}
\usepackage[all]{xy}
\usepackage{esint}
\setlength{\parindent}{0in}
\sisetup{per-mode = symbol}
\usetikzlibrary{calc,arrows,svg.path,decorations.markings,patterns,matrix,3d,fit}
\usepgfplotslibrary{groupplots}
\pgfplotsset{compat=newest}
\newtcolorbox{mydefbox}[2][]{colback=red!5!white,colframe=red!75!black,fonttitle=\bfseries,title=#2,#1}
\newtcolorbox{mythmbox}[2][]{colback=gray!5!white,colframe=gray!75!black,fonttitle=\bfseries,title=#2,#1}
\newtcolorbox{myexamplebox}[2][]{colback=green!5!white,colframe=green!75!black,fonttitle=\bfseries,title=#2,#1}
\newtcolorbox{mypropbox}[2][]{colback=blue!5!white,colframe=blue!75!black,fonttitle=\bfseries,title=#2,#1}
\declaretheoremstyle[headfont=\color{blue}\normalfont\bfseries,]{colored}
\theoremstyle{definition}
\newtheorem{theorem}{Theorem}
\newtheorem{corollary}[theorem]{Corollary}
\newtheorem{lemma}[theorem]{Lemma}
\newtheorem{proposition}[theorem]{Proposition}
\newtheorem{problem}[theorem]{Problem}
\newtheorem{definition}[theorem]{Definition}
\newtheorem{exercise}[theorem]{Exercise}
\newtheorem{example}[theorem]{Example}
\newtheorem{solution}[theorem]{Solution}
\newtheorem*{thm}{Theorem}
\newtheorem*{lem}{Lemma}
\newtheorem*{prob}{Problem}
\newtheorem*{exer}{Exercise}
\newtheorem*{prop}{Proposition}
\def\R{\mathbb{R}}
\def\F{\mathbb{F}}
\def\Q{\mathbb{Q}}
\def\C{\mathbb{C}}
\def\N{\mathbb{N}}
\def\Z{\mathbb{Z}}
\def\Ra{\Rightarrow}
\def\e{\epsilon}
\newcommand{\typo}[1]{{\color{red}{#1}}}
\newcommand\thedate{\today}
\newcommand{\mb}{\textbf}
\newcommand{\norm}[2]{\|{#1}\|_{#2}}
\newcommand{\normm}[1]{\|#1\|}
\newcommand{\mat}[1]{\begin{bmatrix} #1 \end{bmatrix}}
\newcommand{\eqtext}[1]{\hspace{3mm} \text{#1} \hspace{3mm}}
\newcommand{\set}[1]{\{#1\}}
\newcommand{\inte}{\textrm{int}}
\newcommand{\ra}{\rightarrow}
\newcommand{\minv}{^{-1}}
\newcommand{\tx}[1]{\text{ {#1} }}
\newcommand{\abs}[1]{|#1|}
\newcommand{\mc}[1]{\mathcal{#1}}
\newcommand{\uniflim}{\mathop{\mathrm{unif\lim}}}
\newcommand{\notimplies}{\mathrel{{\ooalign{\hidewidth$\not\phantom{=}$\hidewidth\cr$\implies$}}}}
\pagestyle{fancy}
\fancyhf{}
\fancyhead[L]{Title of the Document}
\fancyhead[C]{}
\fancyhead[R]{\thepage}
\fancyfoot[L]{}
\fancyfoot[C]{}
\fancyfoot[R]{}
\renewcommand{\headrulewidth}{0.4pt}
\renewcommand{\footrulewidth}{0.4pt}
\numberwithin{equation}{section}
% Increase spacing between paragraphs
\setlength{\parskip}{1em}
% Increase spacing before and after sections
\usepackage{titlesec}
\titlespacing*{\section}{0pt}{3ex plus 1ex minus .2ex}{2ex plus .2ex}
\titlespacing*{\subsection}{0pt}{2ex plus 1ex minus .2ex}{1ex plus .2ex}
\titlespacing*{\subsubsection}{0pt}{1ex plus 1ex minus .2ex}{1ex plus .2ex}
\title{\textbf{Title of the Document}}
\author{Author Name}
\date{\today}
\begin{document}
\maketitle
\tableofcontents
\newpage
\section{Exercises}
\begin{itemize}
    \item[2.1.] Let $(X, d)$ be a metric space and $S \subset X$. Show that $\overline{S}^\circ = \varnothing$.
    \item[2.2.] Show that for an arbitrary choice of $a, b, c \in \mathbb{R}$, the closed disk $(x - a)^2 + (y - b)^2 \leq r^2$ is a bounded set in $\mathbb{R}^2$.
    \item[2.3.] Let $(X, d)$ be a metric space and let $x, y \in X$. Show that if $d(x, y) < \epsilon$ for every $\epsilon > 0$, then $x = y$.
\end{itemize}

\noindent (2.1) Assume $S \neq \varnothing \Rightarrow \exists \, x \in S, \, \exists \, \epsilon > 0 \, . \, B_\epsilon(x) \subseteq S^\circ$.

$
\text{Then by } x \in S^{\ast\ast}, \, \exists \, \epsilon > 0: B_\frac{\epsilon}{2}(x) \subseteq S_\eta.
$

\noindent However, by $x \in \partial S$, this value of $\epsilon > 0$ implies

$
B_{\frac{\epsilon}{2}}(x) \cap S^\nu \neq \varnothing \Rightarrow B_{\frac{\epsilon}{2}}(x) \not\subseteq S_\eta
$

\noindent which is a contradiction, implying our assumption that $x \not\in \overline{S}^s \cap S^{int}$ must be false and

$
\overline{S}^\circ \cap S^{int} = \varnothing.
$

\vspace{1cm}

\noindent (2.2) \textbf{A set } $\mathcal{S}$ \textbf{is bounded iff } $\exists \, M \in \mathbb{R}^+ \, : \forall x, y \in \mathcal{S} \, d(x, y) \leq M$.

\noindent Let $a, b, r \in \mathbb{R}$.

$
\delta := \left\lbrace (x, y) \in \mathbb{R} \, | \, (x - a)^2 + (y - b)^2 \leq r^2 \right\rbrace
$

\noindent\makebox[\linewidth][c]{
\(
\Rightarrow x^2 - 2ax + a^2 + y^2 - 2by + b^2 \leq{r^2}
\)
}

\noindent\makebox[\linewidth][c]{
\(
\Rightarrow x^2 - 2ax + y^2 - 2yb + r^2 - a^2 - b^2 \leq
\)
}

\noindent\makebox[\linewidth][c]{
\(
\Rightarrow x^2 + y^2 \leq r^2 - a^2 - b^2 + 2ax + 2yb
\)
}

\noindent we need to show $x^2$ is bounded.

\begin{itemize}
    \item $(x - a)^2 \leq r^2$
    \item $\Rightarrow |x - a| \leq |r|$
    \item $\Rightarrow |x - a| \leq |r + a| = |r| + |a|$
    \item $\Rightarrow |x| = |x - a + a| \leq |x - a| + |a| \leq r + |a|$
\end{itemize}

\begin{align*}
    &\text{(diagrams of a circle with radius } a \text{ and a square of side } 2r\text{, situated around the circle)}
\end{align*}

\Rightarrow |y| \leq r + |a| \\
\Rightarrow x^2 \leq (r + |a|)^2

\text{Same for } y_1, \quad y_1^2 \leq (r + |b|)^2 \\

\forall z = (x,y) \in D^2_{a,b}

\|z\| = \sqrt{x^2 + y_2} \\

\leq \sqrt{(r + |a|)^2 + (r + |b|)^2} \\

\text{Thus, if } M = \sqrt{(r + |a|)^2 + (r + |b|)^2}, \text{ the bond holds.}

** \text{Normed boundedness} = \text{distance boundedness.}

\text{Let } x = (x_1, x_2), \quad y = (y_1, y_2) \in D_{a, b} \\

z_3 \in \{x, y\} \\

(z_1 - a)^2 + (z_2 - b)^2 \leq r^2 \\

\Rightarrow d(z, (a, b)) = \sqrt{(z_1 - a)^2 + (z_2 - b)^2} \leq r \\

\Rightarrow d(x, y) \leq d(x, (a, b)) + d(y, (a, b)) \\

&= \sqrt{(x_1 - a)^2 + (x_2 - b)^2} + \sqrt{(y_1 - a)^2 + (y_2 - b)^2} \\

&\leq r + r = 2r.
\

\subsubsection*{(iii)}
Suppose that \( x \ne y \). Then \( d(x, y) \ne 0 \). Thus if we choose \( \varepsilon = d(x, y) \) 
$
\Rightarrow \varepsilon > 0 \text{ but } d(x, y) \notin \varepsilon. 
$
(contradiction).

Suppose \( x \ne y \) and so \( d(x, y) \ne 0 \). Choose \( \varepsilon > 0 \) so that \( \varepsilon = d(x, y) \). Then we must have 
$
d(x, y) < \varepsilon = \frac{d(x, y)}{2},
$
which is a contradiction, as this implies
$
\frac{d(x, y)}{2} \le d(x, y) = \varepsilon, \Rightarrow \varepsilon < \frac{\varepsilon}{2} \Rightarrow 2 \varepsilon < \varepsilon.
$
Thus, \( x = y \).

\subsubsection*{(iv)}
Let \( (V, \| \cdot \|) \) be a normed vsp. Then let \( r > 0 \) and \( x \in V \). Then
$
B_r(x) = \{ u \in V | \ d(x, u) < r \}
$
$
B_{r + \| x \| }(0) = \{ u \in V | \ d(0, u) < r + \| x \| \}
$

$
\includegraphics[width=\textwidth]{Diagram}
$

Let \( y \in B_r(x) \). 
$
d(0, y) \le d(0, x) + d(x, y)
$
$
\le \| x \| + r
$
$
\Rightarrow B_r(x) \subseteq B_{r + \| x \| } (0).
$

\subsubsection*{(v)}
Suppose \( S \) is bounded. Then 
$
\exists M \in \mathbb{R}_{> 0}: \forall x \in S \| x \| \le M.
$
(Equiv to \( \exists M \in \mathbb{R}: \forall x \in V \) \( x \in B_M(0) \))\end{document}