
\documentclass{article}
\usepackage[headheight=20pt, margin=1.0in, top=1.2in]{geometry}
\usepackage{amsmath, amssymb, amsthm, thmtools, tcolorbox, array, graphicx, makeidx, cancel, multirow, fancyhdr, xypic, color, nicefrac, rotating, multicol, caption, subcaption, xcolor, tikz, tikz-3dplot, tikz-cd, pgfplots, import, enumitem, calc, booktabs, wrapfig, siunitx, hyperref,float}
\hypersetup{colorlinks=true,linkcolor=blue}
\usepackage[all]{xy}
\usepackage{esint}
\setlength{\parindent}{0in}
\sisetup{per-mode = symbol}
\usetikzlibrary{calc,arrows,svg.path,decorations.markings,patterns,matrix,3d,fit}
\usepgfplotslibrary{groupplots}
\pgfplotsset{compat=newest}
\newtcolorbox{mydefbox}[2][]{colback=red!5!white,colframe=red!75!black,fonttitle=\bfseries,title=#2,#1}
\newtcolorbox{mythmbox}[2][]{colback=gray!5!white,colframe=gray!75!black,fonttitle=\bfseries,title=#2,#1}
\newtcolorbox{myexamplebox}[2][]{colback=green!5!white,colframe=green!75!black,fonttitle=\bfseries,title=#2,#1}
\newtcolorbox{mypropbox}[2][]{colback=blue!5!white,colframe=blue!75!black,fonttitle=\bfseries,title=#2,#1}
\declaretheoremstyle[headfont=\color{blue}\normalfont\bfseries,]{colored}
\theoremstyle{definition}
\newtheorem{theorem}{Theorem}
\newtheorem{corollary}[theorem]{Corollary}
\newtheorem{lemma}[theorem]{Lemma}
\newtheorem{proposition}[theorem]{Proposition}
\newtheorem{problem}[theorem]{Problem}
\newtheorem{definition}[theorem]{Definition}
\newtheorem{exercise}[theorem]{Exercise}
\newtheorem{example}[theorem]{Example}
\newtheorem{solution}[theorem]{Solution}
\newtheorem*{thm}{Theorem}
\newtheorem*{lem}{Lemma}
\newtheorem*{prob}{Problem}
\newtheorem*{exer}{Exercise}
\newtheorem*{prop}{Proposition}
\def\R{\mathbb{R}}
\def\F{\mathbb{F}}
\def\Q{\mathbb{Q}}
\def\C{\mathbb{C}}
\def\N{\mathbb{N}}
\def\Z{\mathbb{Z}}
\def\Ra{\Rightarrow}
\def\e{\epsilon}
\newcommand{\typo}[1]{{\color{red}{#1}}}
\newcommand\thedate{\today}
\newcommand{\mb}{\textbf}
\newcommand{\norm}[2]{\|{#1}\|_{#2}}
\newcommand{\normm}[1]{\|#1\|}
\newcommand{\mat}[1]{\begin{bmatrix} #1 \end{bmatrix}}
\newcommand{\eqtext}[1]{\hspace{3mm} \text{#1} \hspace{3mm}}
\newcommand{\set}[1]{\{#1\}}
\newcommand{\inte}{\textrm{int}}
\newcommand{\ra}{\rightarrow}
\newcommand{\minv}{^{-1}}
\newcommand{\tx}[1]{\text{ {#1} }}
\newcommand{\abs}[1]{|#1|}
\newcommand{\mc}[1]{\mathcal{#1}}
\newcommand{\uniflim}{\mathop{\mathrm{unif\lim}}}
\newcommand{\notimplies}{\mathrel{{\ooalign{\hidewidth$\not\phantom{=}$\hidewidth\cr$\implies$}}}}
\pagestyle{fancy}
\fancyhf{}
\fancyhead[L]{Title of the Document}
\fancyhead[C]{}
\fancyhead[R]{\thepage}
\fancyfoot[L]{}
\fancyfoot[C]{}
\fancyfoot[R]{}
\renewcommand{\headrulewidth}{0.4pt}
\renewcommand{\footrulewidth}{0.4pt}
\numberwithin{equation}{section}
% Increase spacing between paragraphs
\setlength{\parskip}{1em}
% Increase spacing before and after sections
\usepackage{titlesec}
\titlespacing*{\section}{0pt}{3ex plus 1ex minus .2ex}{2ex plus .2ex}
\titlespacing*{\subsection}{0pt}{2ex plus 1ex minus .2ex}{1ex plus .2ex}
\titlespacing*{\subsubsection}{0pt}{1ex plus 1ex minus .2ex}{1ex plus .2ex}
\title{\textbf{Title of the Document}}
\author{Author Name}
\date{\today}
\begin{document}
\maketitle
\tableofcontents
\newpage
\section{Exercises}

\begin{enumerate}
    \item[2.1.] Let $(X, d)$ be a metric space and $S \subset X$. Show that $\partial S \subset S^{int} = \emptyset$.
    \item[2.2.] Show that for an arbitrary choice of $a, b, r \in \mathbb{R}$, the closed disk $(x - a)^2 + (y - b)^2 \leq r^2$ is in a bounded set in $\mathbb{R}^2$.
    \item[2.3.] Let $(X, d)$ be a metric space and let $x, y \in X$. Show that if $d(x, y) < \epsilon$ for every $\epsilon > 0$, then $x = y$.
\end{enumerate}

\begin{proof}[2.1]
Assume $\partial S \subset S^{int}$. Then $\exists x \in S^{int} \subset \partial S$. Then by $x \in S^{int} \implies \exists \epsilon > 0 : B_\epsilon(x) \subset S$.

However, by $x \in \partial S$, this value of $\epsilon > 0$ implies $B_{\frac{\epsilon}{2}}(x) \cap S^c \neq \emptyset \implies B_{\frac{\epsilon}{2}}(x) \not\subset S$, which is a contradiction, implying our assumption that $x \in \partial S \cap S^{int}$ must be false and $\partial S \cap S^{int} = \emptyset$.
\end{proof}

\begin{proof}[2.2]
A set $S$ is bounded iff $\exists M \in \mathbb{R}^+ : \forall x, y \in S \ d(x, y) \leq M$.

Let $a, b, r \in \mathbb{R}$. 

$
S := \{ (x, y) \in \mathbb{R}^2 \ | \ (x - a)^2 + (y - b)^2 \leq r^2 \}
$

$
\implies x^2 - 2ax + a^2 + y^2 - 2yb + b^2 \leq r^2
$

$
\implies x^2 - 2ax + y^2 - 2yb \leq r^2 - a^2 - b^2
$

$
\implies x^2 + y^2 \leq r^2 - a^2 - b^2 + 2ax + 2yb
$

We need to show $x^2$ is bounded. 

\begin{itemize}
    \item $(x - a)^2 \leq r^2$
    \item $\implies |x - a| \leq |r|$
    \item $\implies |x - a| = |x| + |a| \leq |r| + |a|$
    \item $\implies |x| = |x - a + a| \leq |x - a| + |a| \leq r + |a|$
\end{itemize}
\end{proof}

\begin{align*}
&\implies |y| \leq r + |a| \\
&\implies y^2 \leq (r+|a|)^2
\end{align*}

Same for $x$, 

$ x^2 \leq (r + |b|)^2 $

For $z = (x,y) \in D^2_{a,b}$

$ \| z \| = \sqrt{x^2 + y^2} $

$
\leq \sqrt{(r + |a|)^2 + (r + |b|)^2}
$

Thus if 
$
M = \sqrt{(r + |a|)^2 + (r + |b|)^2}
$
the bound holds.

\# IS named boundness = distance boundness.

Let \( x = (x_1, x_2), y = (y_1, y_2) \in D_{a,b} \)

$ z_1, z_2 \in \{x,y\} $

$
(z_2 - a)^2 + (z_2 - b)^2 = r^2
$

$
\implies d(z_i, (a, b)) = \sqrt{(z_1 - a)^2 + (z_2 - b)^2} \leq r 
$

$
\implies d(x,y) \leq d(x, (a,b)) + d(y, (a,b))
$

$
= \sqrt{(x_1 - a)^2 + (x_2 - b)^2} + \sqrt{(y_1 - a)^2 + (y_2 - b)^2}
$

$
\leq r + r = 2r.
$

\begin{enumerate}
    \item[(iii)] Suppose that \( x \neq y \). Then \( d(x,y) \neq 0 \). Thus if we choose \(\epsilon = d(x,y)\) implies that \( \epsilon > 0 \) but \( d(x,y) \notin \epsilon \). (contradiction).

    \item[Contradiction] Suppose \( x = y \) and so \( d(x,y) = 0 \). Choose \(\epsilon > 0 \) so that \(\epsilon = d(x,y) \). Then we must have \( d(x,y) < \epsilon = \frac{d(x,y)}{2} \), which is a contradiction, as this implies if \( d(x,y) < \epsilon \implies d(x,y) = s < \epsilon = \frac{s}{2} \).

    \(-s < s/2 \implies 2s < s \).

    Thus \( x \neq y \).

    \item[(iv)] Let \( (V, || \cdot || )\) be a normed vsp. Then let \( r > 0 \) and \( x \in V \). Then
    $
    B_{r}(x) = \{ u \in V | d(x,u) < r \}
    $
    $
    B_{\epsilon + ||x||}(0) = \{ v \in V | d(0,v) < r + ||x|| \}
    $
    $
    \text{Let } y \in B_r(x). 
    $
    $
    d(0,y) \leq d(0,x) + d(x,y)
    $
    $
    \leq ||x|| + r
    $
    $
    \implies B_r(x) \subseteq B_{\epsilon + ||x||}(0).
    $

    \item[(v)] Suppose \( S \) is bounded. Then \(\exists M \epsilon \mathbb{R}: \forall x \epsilon S \ ||x|| \leq M \).
    $
    (Equivalent to\ \exists M \epsilon \mathbb{R}: \forall x \epsilon v) x \epsilon B_{M}(0)
    $
\end{enumerate}\end{document}