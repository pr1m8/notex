
\documentclass{article}
\usepackage[headheight=20pt, margin=1.0in, top=1.2in]{geometry}
\usepackage{amsmath, amssymb, amsthm, thmtools, tcolorbox, array, graphicx, makeidx, cancel, multirow, fancyhdr, xypic, color, nicefrac, rotating, multicol, caption, subcaption, xcolor, tikz, tikz-3dplot, tikz-cd, pgfplots, import, enumitem, calc, booktabs, wrapfig, siunitx, hyperref,float}
\hypersetup{colorlinks=true,linkcolor=blue}
\usepackage[all]{xy}
\usepackage{esint}
\setlength{\parindent}{0in}
\sisetup{per-mode = symbol}
\usetikzlibrary{calc,arrows,svg.path,decorations.markings,patterns,matrix,3d,fit}
\usepgfplotslibrary{groupplots}
\pgfplotsset{compat=newest}
\newtcolorbox{mydefbox}[2][]{colback=red!5!white,colframe=red!75!black,fonttitle=\bfseries,title=#2,#1}
\newtcolorbox{mythmbox}[2][]{colback=gray!5!white,colframe=gray!75!black,fonttitle=\bfseries,title=#2,#1}
\newtcolorbox{myexamplebox}[2][]{colback=green!5!white,colframe=green!75!black,fonttitle=\bfseries,title=#2,#1}
\newtcolorbox{mypropbox}[2][]{colback=blue!5!white,colframe=blue!75!black,fonttitle=\bfseries,title=#2,#1}
\declaretheoremstyle[headfont=\color{blue}\normalfont\bfseries,]{colored}
\theoremstyle{definition}
\newtheorem{theorem}{Theorem}
\newtheorem{corollary}[theorem]{Corollary}
\newtheorem{lemma}[theorem]{Lemma}
\newtheorem{proposition}[theorem]{Proposition}
\newtheorem{problem}[theorem]{Problem}
\newtheorem{definition}[theorem]{Definition}
\newtheorem{exercise}[theorem]{Exercise}
\newtheorem{example}[theorem]{Example}
\newtheorem{solution}[theorem]{Solution}
\newtheorem*{thm}{Theorem}
\newtheorem*{lem}{Lemma}
\newtheorem*{prob}{Problem}
\newtheorem*{exer}{Exercise}
\newtheorem*{prop}{Proposition}
\def\R{\mathbb{R}}
\def\F{\mathbb{F}}
\def\Q{\mathbb{Q}}
\def\C{\mathbb{C}}
\def\N{\mathbb{N}}
\def\Z{\mathbb{Z}}
\def\Ra{\Rightarrow}
\def\e{\epsilon}
\newcommand{\typo}[1]{{\color{red}{#1}}}
\newcommand\thedate{\today}
\newcommand{\mb}{\textbf}
\newcommand{\norm}[2]{\|{#1}\|_{#2}}
\newcommand{\normm}[1]{\|#1\|}
\newcommand{\mat}[1]{\begin{bmatrix} #1 \end{bmatrix}}
\newcommand{\eqtext}[1]{\hspace{3mm} \text{#1} \hspace{3mm}}
\newcommand{\set}[1]{\{#1\}}
\newcommand{\inte}{\textrm{int}}
\newcommand{\ra}{\rightarrow}
\newcommand{\minv}{^{-1}}
\newcommand{\tx}[1]{\text{ {#1} }}
\newcommand{\abs}[1]{|#1|}
\newcommand{\mc}[1]{\mathcal{#1}}
\newcommand{\uniflim}{\mathop{\mathrm{unif\lim}}}
\newcommand{\notimplies}{\mathrel{{\ooalign{\hidewidth$\not\phantom{=}$\hidewidth\cr$\implies$}}}}
\pagestyle{fancy}
\fancyhf{}
\fancyhead[L]{Title of the Document}
\fancyhead[C]{}
\fancyhead[R]{\thepage}
\fancyfoot[L]{}
\fancyfoot[C]{}
\fancyfoot[R]{}
\renewcommand{\headrulewidth}{0.4pt}
\renewcommand{\footrulewidth}{0.4pt}
\numberwithin{equation}{section}
% Increase spacing between paragraphs
\setlength{\parskip}{1em}
% Increase spacing before and after sections
\usepackage{titlesec}
\titlespacing*{\section}{0pt}{3ex plus 1ex minus .2ex}{2ex plus .2ex}
\titlespacing*{\subsection}{0pt}{2ex plus 1ex minus .2ex}{1ex plus .2ex}
\titlespacing*{\subsubsection}{0pt}{1ex plus 1ex minus .2ex}{1ex plus .2ex}
\title{\textbf{Title of the Document}}
\author{Author Name}
\date{\today}
\begin{document}
\maketitle
\tableofcontents
\newpage
\section{Interior and Boundary Points}

\begin{mydefbox}{Definition 2.1}
Let $(X,d)$ be a metric space. If $ x \in X $ and $ r > 0 $, we define the \textbf{open ball of radius r centred at x} as 
$ 
B_{r}(x) := \{ y \in X : d(x,y) < r \}. 
$
\end{mydefbox}

In $\mathbb{R}^{n}$ with the Euclidean metric $d(x,y) = || x - y || $, the open ball $B_{r}(x)$ is nothing more than the collection of points which are a distance at most $r$ from $x$. This generalizes the interval, since in $\mathbb{R}^{1}$ we have 
$ 
B_{r}(x) = \{ y \in \mathbb{R} : |x - y| < r \} = (x - r, x + r), 
$
or if we centre around $0$, $ B_{r}(0) = (-r, r)$. In $\mathbb{R}^{2} $ we get a disk of radius $r,$ 
$ 
B_{r}(0) = \{ (x,y) \in \mathbb{R}^{2} : \sqrt{x^{2} + y^{2}} < r \}, 
$
which we recognize as being the same as $x^{2} + y^{2} < r^{2}.$

\begin{center}
\begin{tikzpicture}
    % Draw circle
    \draw (0,0) circle (2);

    % Label x
    \node at (-0.2,-0.2) {$x$};
    \fill (0,0) circle (2pt);

    % Label radius r
    \draw[<->] (0,0) -- (2,0);
    \node at (1,0.2) {$r$};

    % Label B_r(x)
    \node at (2.5,1.5) {$B_r(x)$};
\end{tikzpicture}
\end{center}

\begin{center}
\textbf{Figure 2.1:} In $\mathbb{R}^{2}$, the open ball of radius $r$ centred at $x$ consists of all points which are a distance at most $r$ from $x$.
\end{center}

\begin{mydefbox}{Definition 2.2}
A subspace $S$ of a metric space $X$ is \textbf{bounded} if there exists an $r > 0$ and an $a \in X$ such that $S \subseteq B_{r}(a)$.
\end{mydefbox}

\begin{mydefbox}{Definition 2.3}
Let $(X,d)$ be a metric space, and $S \subseteq \mathbb{R}^{n}$ be an arbitrary set.
\begin{enumerate}
    \item We say that $x \in S$ is an \textbf{interior point of S} if there exists an $r > 0$ such that $B_{r}(x) \subseteq S$; that is, $x$ is an interior point if we can enclose it in an open ball which is entirely contained in $S$.
    \item We say that $x \in S$ is a \textbf{boundary point of S} if for every $r > 0$, $B_{r}(x) \cap S \not= \emptyset$ and $B_{r}(x) \cap S^{c} \not= \emptyset$; that is, $x$ is a boundary point if no matter what ball we place around $x$, that ball lives both inside and outside of $S$. 
\end{enumerate}
\end{mydefbox}

The \textbf{interior of $S$} - denoted $\mathring{S}$ - is the collection of interior points of $S$, while \textbf{boundary of $S$} - denoted $\partial S$ - is the collection of boundary points of $S$.

\begin{center}
\begin{tikzpicture}
    % Draw circle S
    \draw (0,0) ellipse (3 and 2);

    % Label S
    \node at (3.5,2) {$S$};

    % Label points
    \fill (-1,0) circle (2pt);
    \node at (-1.2,0.3) {$p$};

    \fill (2,-1) circle (2pt);
    \node at (2.2,-0.7) {$q$};

    % Draw circles around p and q
    \draw[dashed] (-1,0) circle (1);
    \draw[dashed] (2,-1) circle (0.5);
\end{tikzpicture}
\end{center}

\begin{center}
\textbf{Figure 2.2:} The point $q$ is a boundary point. No matter what size ball we place around $q$, that ball will intersect both $S$ and $S^{c}$. On the other hand, $p$ is an interior point, since we can place a ball around it which lies entirely within $S$.
\end{center}

We should take a moment and think about these definitions, and why they make sense. A boundary point is any point at which occurs at the very fringe of the set; that is, if I push a little further I will leave the set. An interior point should be a point inside of $S$, such that if I move in any direction a sufficiently small distance, I stay within the set. Note that if $x$ is an interior point then we must have that $x \in S$; however, boundary points do not need to be in the set. We start with a simple example.

\begin{myexamplebox}
\textbf{Example 2.4}
Let \( S = (-1,1) \subseteq \mathbb{R} \) endowed with the Euclidean metric. What are the interior points and the boundary points of S?
\end{myexamplebox}

\textbf{Solution.} I claim that any point in \((-1,1)\) is an interior point. To see that this is the case, let \( p \in (-1,1) \) be an arbitrary point. We need to place a ball around \( p \) which lies entirely within \((-1, 1)\).

\medskip

To do this, assume without loss of generality that \( p \geq 0 \). If \( p = 0 \) then we can set \( r = 1/2 \) and \( \mathbf{B}_{r}(p) = (-1/2, 1/2) \subseteq (-1, 1) \). Thus assume that \( p \neq 0 \) and let \( r = (1-p)/2 \), which represents half the distance from \( p \) to 1. I claim that \( \mathbf{B}_{r}(p) \subseteq (-1,1) \). Indeed, let \( x \in \mathbf{B}_{r}(p) \) be any point, so that \( | x - p | < r \) by definition. Then
$
| x | = | x - p + p | \leq | x - p | + | p | < r + p = \frac{1 - p}{2} + p = \frac{1 + p}{2} < 1
$
where in the last inequality we have used the fact that \( p < 1 \) so \( 1 + p < 2 \). Thus \( x \in (-1,1) \), and since \( x \) was arbitrary, \( \mathbf{B}_{r}(p) \subseteq (-1,1) \).

The boundary points are \(\pm 1\), where we note that even though \( -1 \notin (-1,1) \), it is still a boundary point. To see that \(-1 \) is a boundary point, let \( r > 0 \) be arbitrary, so that \( \mathbf{B}_{r}(p) = ( -1 - r, 1 + r) \). We then have \( \mathbf{B}_{r}(p) \cap (-1,1) = ( -1 - r, 1) \neq \emptyset \),and \( \mathbf{B}_{r}(p) \cap (-1, 1)^{c} = ( -1 - r, 1) \neq \emptyset \), as required. The proof for \( -1 \) is analogous and left as an exercise.
\qed

\begin{myexamplebox}
\textbf{Example 2.5}
What is the boundary of \(\mathbb{Q}\) in \(\mathbb{R}\) with the Euclidean metric?
\end{myexamplebox}

\textbf{Solution.} We claim that \(\partial \mathbb{Q} = \mathbb{R} \). Since both the irrationals and the rationals are dense in the real numbers, we know that every non-empty open interval in \(\mathbb{R}\) contains both a rational and irrational number. Thus let \( x \in \mathbb{R} \) be any real number, and \( r > 0 \) be arbitrary. The set \( \mathbf{B}_{r}(x) \) is an open interval around \( x \), and contains a rational number, showing that \( \mathbf{B}_{r}(x) \cap \mathbb{Q} \neq \emptyset \). Similarly, \( \mathbf{B}_{r}(x) \) contains an irrational number, showing that \( \mathbf{B}_{r}(x) \cap \mathbb{Q}^{c} \neq \emptyset \), so \( x \in \partial \mathbb{Q} \). Since \( x \) was arbitrary, we conclude that \( \partial \mathbb{Q} = \mathbb{R} \).
\qed

\section{Open and Closed Sets}

\begin{mydefbox}{Definition 2.6}
A set $S$ in a metric space $(X, d)$ is said to be \textbf{open} if every point of $S$ is an interior point; that is, $S$ is open if for every $x \in S$ there exists an $r > 0$ such that $B_d(x; r) \subseteq S$. The set $S$ is \textbf{closed} if $S$ is open. Given a point $x \in X$, an open neighbourhood of $x$ is some open set containing $x$.
\end{mydefbox}

\begin{myexamplebox}{Example 2.7}
The set $S = \{(x, y) \in \mathbb{R}^2 : y > 0\} \subseteq \mathbb{R}^2$ is open in the Euclidean metric.
\end{myexamplebox}

\begin{center}
\section{Open, Closed, and Everything in Between}
\section{The Topology of $\mathbb{R}^n$}
\end{center}

\begin{center}
\includegraphics[scale=0.5]{open_set_diagram.png} \\
\textbf{Figure 2.3}: The upper half-plane is open. For any point, look at its y-coordinate $p_y$ and use the ball of radius $p_y /2$. 
\end{center}

\textbf{Solution.} We need to show that around every point in $S$ we can place an open ball that remains entirely within $S$. Choose a point $p = (p_x, p_y) \in S$, so that $p_y > 0$, and let $r = p_y / 2$. Consider the ball $B_r (p)$, which we claim lives entirely within $S$. To see that this is the case, choose any point $q = (q_x, q_y) \in B_r (p)$. Now

$ p_y - q_y \leq \|q - p\| < r = \frac{p_y}{2} $

which implies that $q_y > p_y - p_y / 2 = p_y / 2 > 0$. Since $q_y > 0$, this shows that $q \in S$, and since $q$ was arbitrary, $B_r (p) \subseteq S$ as required. \hspace*{\fill} $\square$\end{document}